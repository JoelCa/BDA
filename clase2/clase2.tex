\documentclass{fancyslides} 
\usepackage[utf8]{inputenc}
\usepackage{times}


\graphicspath{{img/}}

%%% Beamer settings (do not change)
\usetheme{default} 
\setbeamertemplate{navigation symbols}{} %no navigation symbols
\setbeamercolor{structure}{fg=\yourowntexcol} 
\setbeamercolor{normal text}{fg=\yourowntexcol} 



%%%%%%%%%%%%%%%%%%%%%%%%%
%%% CUSTOMISATIONS %%%%%%
%%%%%%%%%%%%%%%%%%%%%%%%%


%%%% SLIDE ELEMENTS
\newcommand{\structureopacity}{0.75} %opacity for the structure elements (boxes and dots)
\newcommand{\strcolor}{black} %elements colour (predefined blue; orange; green)

%%%% TEXT COLOUR
\newcommand{\yourowntexcol}{white}


%%%%%%%%%%%%%%%%%%%%%%%%%
%%% TITLE SLIDE DATA %%%%
%%%%%%%%%%%%%%%%%%%%%%%%%
\fbckg{abstract-data}
\newcommand{\titlephrase}{On-Line Analytical Processing}
\newcommand{\name}{Javier Bonet \\ Joel Catacora \\}
\newcommand{\affil}{Base de datos avanzada}
\newcommand{\email}{22 de abril del 2015}


\begin{document}


\startingslide %this generates titlepage from the data above

\fbckg{white}
\begin{frame}
\pointedsl{{\LARGE ¿Qué es OLAP?}}
\end{frame}


\fbckg{white}
\begin{frame}
\misc
{
El \textbf{procesamiento analítico en línea} (OLAP) es una solución utilizada en el campo de la inteligencia de negocios, cuyo objetivo es permitir la consulta de grandes cantidades de datos de forma eficiente y sencilla.
}
\end{frame}

\fbckg{white}
\begin{frame}
\misc
{
El concepto OLAP puede ser definido con solo 5 palabras: Análisis Rápido de Información Compartida Multidimensional, (Fast Analysis of Shared Multidimensional Information, o FASMI).
}
\end{frame}

\fbckg{white}
\begin{frame}
\misc
{
\begin{itemize}
  \item \textbf{Rápida}: el sistema está dirigido a proporcionar la mayoría de las respuestas a los usuarios en pocos segundos.
  \item \textbf{Análisis}: el sistema puede hacer frente a cualquier lógica de negocio y análisis estadístico que sea relevante para el usuario, en forma relativamente sencilla.
  \item \textbf{Compartida}: significa que el sistema implementa todos los requisitos de seguridad de la confidencialidad.
  \item \textbf{Multidimensionalidad}: el sistema debe proveer una vista conceptual multidimensional de los datos.
  \item \textbf{Información}: se refiere a todos los datos y la información derivada, que sea relevante para la aplicación.
\end{itemize}
}
\end{frame}



\fbckg{white}
\begin{frame}
\pointedsl{{\LARGE Características}}
\end{frame}


\fbckg{white}
\begin{frame}
\itemized{
\item \textbf{Visión multidimensional de los datos}
}
\end{frame}

\fbckg{white}
\begin{frame}
\itemized{
\item \textbf{Visión multidimensional de los datos}
\item \textbf{Soporta cálculos complejos}
}
\end{frame}

\fbckg{white}
\begin{frame}
\itemized{
\item \textbf{Visión multidimensional de los datos}
\item \textbf{Soporta cálculos complejos}
\item \textbf{Inteligencia respecto del tiempo}
}
\end{frame}

\fbckg{white}
\begin{frame}
\pointedsl{\LARGE{Agregaciones}}
\end{frame}

\fbckg{white}
\begin{frame}
\misc
{
Los \textbf{agregados} se utilizan en los modelos dimensionales del Data Warehouse para producir efectos positivos sobre el tiempo que toma una consulta sobre grandes conjuntos de datos. El uso más común de los agregados es tomar una dimensión y cambiar su granularidad.
Al cambiar la granularidad de la dimensión, la tabla de hechos tiene que ser parcialmente resumida para adaptarse a la nueva dimensión, creando así nuevas tablas dimensionales y de hecho, que encajan en este nuevo nivel de granularidad. 
}
\end{frame}

\fbckg{white}
\begin{frame}
\pointedsl{\LARGE{Hipercubo de datos}}
\end{frame}

\fbckg{white}
\begin{frame}
\misc
{
El cubo OLAP puede ser pensado como una extensión de la matriz multidimensional de una hoja de cálculo, de ahí el nombre del hipercubo. Técnicamente, el cubo de datos es una representación multidimensional de datos, junto con todos los agregados posible, es decir, los agregados que resultan mediante la selección de un subconjunto propio de las dimensiones y sumando sobre todas las dimensiones restantes.
}
\end{frame}

\end{document}
