\documentclass{fancyslides} 
\usepackage[utf8]{inputenc}
\usepackage{times}

\usepackage{graphicx}
\usepackage{caption}
\usepackage{subcaption}
\usepackage{ragged2e}


\graphicspath{{img/}}

\addtobeamertemplate{block begin}{}{\justifying}  %new code

%%% Beamer settings (do not change)
\usetheme{default} 
\setbeamertemplate{navigation symbols}{} %no navigation symbols
\setbeamercolor{structure}{fg=\yourowntexcol} 
\setbeamercolor{normal text}{fg=\yourowntexcol} 



%%%%%%%%%%%%%%%%%%%%%%%%%
%%% CUSTOMISATIONS %%%%%%
%%%%%%%%%%%%%%%%%%%%%%%%%


%%%% SLIDE ELEMENTS
\newcommand{\structureopacity}{0.75} %opacity for the structure elements (boxes and dots)
\newcommand{\strcolor}{orange} %elements colour (predefined blue; orange; green)

%%%% TEXT COLOUR
\newcommand{\yourowntexcol}{white}


%%%%%%%%%%%%%%%%%%%%%%%%%
%%% TITLE SLIDE DATA %%%%
%%%%%%%%%%%%%%%%%%%%%%%%%
\fbckg{fondo}
\newcommand{\titlephrase}{Herramientas y temáticas actuales de Data Warehouses y OLAP}
\newcommand{\name}{Javier Bonet \\ Joel Catacora \\}
\newcommand{\affil}{Base de datos avanzada}
\newcommand{\email}{13 de mayo del 2015}


\begin{document}


\startingslide %this generates titlepage from the data above

\fbckg{white}
\begin{frame}
\pointedsl{{\Large Diseño general de un DW}}
\end{frame}

\fbckg{white}
\begin{frame}
\misc
{
\begin{block}{} %lo uso para justificar el texto
 Este método se basa en la suposición de que los DWs son un tipo particular de bases de datos dedicado para fines analíticos. Por lo tanto, su diseño debe seguir las fases de diseño de bases de datos tradicionales, es decir, la especificación de los requisitos, diseño conceptual, diseño lógico y diseño físico, como se muestra en la figura. Sin embargo, hay diferencias significativas entre las fases del diseño de bases de datos y almacenes de datos, que se derivan de su diferente naturaleza.
\end{block}
\begin{center}
\includegraphics[scale=0.25]{diseno1}  
\end{center}

}
\end{frame}

\fbckg{white}
\begin{frame}
\misc
{
\begin{itemize}
  \item \textbf{Especificación de los requerimientos}: determina, entre otras cosas, \textit{que} datos deben estar disponibles y \textit{cómo} deben organizarse. 
  \item \textbf{Diseño conceptual}: la fase anterior debe proporcionar los elementos necesarios para la construcción de un esquema conceptual inicial del Data Warehouse.
  \item \textbf{Diseño lógico}: primero, consiste en la transformación del esquema conceptual en un esquema lógico; y segundo, la especificación de los procesos ETL.
  \item \textbf{Diseño físico}: abarca la implementación, tanto de los esquema del DW, como de los procesos ETL. Durante la fase de diseño físico, el esquema lógico se convierte en una estructura de base de datos física.
\end{itemize}


}
\end{frame}


\end{document}
