\documentclass{fancyslides} 
\usepackage[utf8]{inputenc}
\usepackage{times}

\graphicspath{{img/}}

%%% Beamer settings (do not change)
\usetheme{default} 
\setbeamertemplate{navigation symbols}{} %no navigation symbols
\setbeamercolor{structure}{fg=\yourowntexcol} 
\setbeamercolor{normal text}{fg=\yourowntexcol} 



%%%%%%%%%%%%%%%%%%%%%%%%%
%%% CUSTOMISATIONS %%%%%%
%%%%%%%%%%%%%%%%%%%%%%%%%


%%%% SLIDE ELEMENTS
\newcommand{\structureopacity}{0.75} %opacity for the structure elements (boxes and dots)
\newcommand{\strcolor}{blue} %elements colour (predefined blue; orange; green)

%%%% TEXT COLOUR
\newcommand{\yourowntexcol}{white}


%%%%%%%%%%%%%%%%%%%%%%%%%
%%% TITLE SLIDE DATA %%%%
%%%%%%%%%%%%%%%%%%%%%%%%%
\fbckg{nexus}
\newcommand{\titlephrase}{Data Warehouse}
\newcommand{\name}{Javier Bonet \\ Joel Catacora \\}
\newcommand{\affil}{Base de datos avanzada}
\newcommand{\email}{1 de abril del 2015}


\begin{document}


\startingslide %this generates titlepage from the data above

\fbckg{negro}
\begin{frame}
\pointedsl{Business intelligence}
\end{frame}

\fbckg{negro}
\begin{frame}
\misc{El término "Business Intelligence" fue acuñado originalmente por Richard Millar Devens en 'Cyclopædia of Commercial and Business Anecdotes'en 1865. Devens utilizó el término para describir cómo el banquero, Sir Henry Furnese, obtuvo beneficios por recibir y procesar la información sobre su entorno, antes que sus competidores.

La inteligencia de negocios, tal como se entiende, hoy en día se dice que ha evolucionado desde los sistemas de apoyo a las decisiones (DSS), que se inició en la década de 1960 y desarrolló a mediado de los años 80's.
}
\end{frame}

\fbckg{negro}
\begin{frame}
\misc{Se denominada \textbf{Business intelligence} (BI), al conjunto de estrategias que integran, por un lado el almacenamiento, y
por el otro, el procesamiento de grandes cantidades de datos, con el principal objetivo de
transformarlos en conocimiento y en decisiones en tiempo real, a través del análisis de los datos existentes en una organización o empresa.

\begin{center}
Dato + Análisis = Conocimiento.
\end{center}
}
\end{frame}


\fbckg{negro}
\begin{frame}
\pointedsl{¿Qué es DW?}
\end{frame}


% \fbckg{negro}
% \begin{frame}
% \misc{\textbf{Data Warehouse} (DW o DWH), es una colección de datos orientada a un ámbito determinado, integrado, no volátil y variable en el tiempo, que ayuda a la toma de decisiones en la entidad en la que se utiliza. Se trata, sobre todo, de un expediente completo de una organización, más allá de la información transaccional y operacional. Almacenado en una base de datos diseñada para favorecer el análisis y la divulgación eficiente de datos.}
% \end{frame}

\fbckg{negro}
\begin{frame}
\misc{Un \textbf{Data Warehouse} (DW o DWH), es un sistema que extrae, limpia, ajusta, y entrega los
datos de origen en un almacén de datos dimensional, y luego apoya e implementa
consultas y análisis, con el propósito de la toma de decisiones.}
\end{frame}

\fbckg{negro}
\begin{frame}
\pointedsl{Componentes}
\end{frame}

\fbckg{negro}
\begin{frame}
\misc{\includegraphics[width=0.9\linewidth]{componentes}}
\end{frame}

\fbckg{negro}
\begin{frame}
\pointedsl{Características}
\end{frame}

\fbckg{negro}
\begin{frame}
\itemized{
\item \textbf{Integrado}: Se integran datos provenientes de múltiples fuentes, posiblemente distintas.
}
\end{frame}

\fbckg{negro}
\begin{frame}
\itemized{
\item \textbf{Integrado}: Se integran datos provenientes de múltiples fuentes, posiblemente distintas.
\item \textbf{No volátil}: Una vez almacenados los datos en el DW, la información que éstos representan no debe perderse.
}
\end{frame}

\fbckg{negro}
\begin{frame}
\itemized{
\item \textbf{Integrado}: Se integran datos provenientes de múltiples fuentes, posiblemente distintas.
\item \textbf{No volátil}: Una vez almacenados los datos en el DW, la información que éstos representan no debe perderse.
\item \textbf{Variable en el tiempo}: La información histórica se mantiene en el DW a lo largo del tiempo.
}
\end{frame}


% \fbckg{blank}
% \begin{frame}
% \framedsl{\pitem{pointed slogan} \pitem{framed slogan} \fitem{beamer features}}
% \end{frame}
% 

% \fbckg{blank}
% \begin{frame}
%   \thankyou   %%%% ending slide with thank you notice
% \end{frame}


% \fbckg{blank}
% \begin{frame}
% \sources{
% \includegraphics[scale=0.048]{1} \ flickr/lovelornpoets\\
% \includegraphics[scale=0.2]{2} \ flickr/apsmuseum
% }
% \end{frame}

\end{document}
