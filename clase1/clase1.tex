\documentclass{fancyslides} 
\usepackage[utf8]{inputenc}
\usepackage{times}


\graphicspath{{img/}}

%%% Beamer settings (do not change)
\usetheme{default} 
\setbeamertemplate{navigation symbols}{} %no navigation symbols
\setbeamercolor{structure}{fg=\yourowntexcol} 
\setbeamercolor{normal text}{fg=\yourowntexcol} 



%%%%%%%%%%%%%%%%%%%%%%%%%
%%% CUSTOMISATIONS %%%%%%
%%%%%%%%%%%%%%%%%%%%%%%%%


%%%% SLIDE ELEMENTS
\newcommand{\structureopacity}{0.75} %opacity for the structure elements (boxes and dots)
\newcommand{\strcolor}{blue} %elements colour (predefined blue; orange; green)

%%%% TEXT COLOUR
\newcommand{\yourowntexcol}{white}


%%%%%%%%%%%%%%%%%%%%%%%%%
%%% TITLE SLIDE DATA %%%%
%%%%%%%%%%%%%%%%%%%%%%%%%
\fbckg{nexus}
\newcommand{\titlephrase}{Data Warehouse}
\newcommand{\name}{Javier Bonet \\ Joel Catacora \\}
\newcommand{\affil}{Base de datos avanzada}
\newcommand{\email}{1 de abril del 2015}


\begin{document}


\startingslide %this generates titlepage from the data above

\fbckg{negro}
\begin{frame}
\pointedsl{{\LARGE Business intelligence}}
\end{frame}

\fbckg{negro}
\begin{frame}
\misc{%El término "\textbf{Business Intelligence}" fue acuñado originalmente por Richard Millar Devens en 'Cyclopædia of Commercial and Business Anecdotes'en 1865. Devens utilizó el término para describir cómo el banquero, Sir Henry Furnese, obtuvo beneficios por recibir y procesar la información sobre su entorno, antes que sus competidores.
% \newline
% 
% La inteligencia de negocios, tal como se entiende, hoy en día se dice que ha evolucionado desde los sistemas de apoyo a las decisiones (DSS), que se inició en la década de 1960 y desarrolló a mediado de los años 80's.

En 1989, Howard Dresner (un analista de Gartner Group) propuso la "\textbf{inteligencia de negocios}" como un término general para describir "los conceptos y métodos para mejorar
la toma de decisiones empresariales mediante el uso de sistemas basados en echos".
}
\end{frame}

\fbckg{negro}
\begin{frame}
\misc{Se denominada \textbf{Business intelligence} (BI), al conjunto de estrategias que integran, por un lado el almacenamiento, y
por el otro, el procesamiento de grandes cantidades de datos, con el principal objetivo de
transformarlos en conocimiento y en decisiones en tiempo real, a través del análisis de los datos existentes en una organización o empresa.

\begin{center}
Dato + Análisis = Conocimiento.
\end{center}
}
\end{frame}


\fbckg{negro}
\begin{frame}
\pointedsl{{\large ¿Qué es un Data Warehouse?}}
\end{frame}


\fbckg{negro}
\begin{frame}
\misc{
Definición en términos de las características del DW:
\newline

\textbf{Data Warehouse} (DW o DWH), es una colección de datos orientada a un ámbito determinado, integrado, no volátil y variable en el tiempo, que ayuda a la toma de decisiones en la entidad en la que se utiliza.
}
\end{frame}

\fbckg{negro}
\begin{frame}
\misc{
Una definición más amplia que la anterior:
\newline

Un \textbf{Data Warehouse}, es un sistema que extrae, limpia, ajusta, y entrega los
datos de origen en un almacén de datos dimensional, y luego apoya e implementa
consultas y análisis, con el propósito de asistir en la toma de decisiones.}
\end{frame}

\fbckg{negro}
\begin{frame}
\pointedsl{Características}
\end{frame}

\fbckg{negro}
\begin{frame}
\itemized{
\item \textbf{Integrado}: Se integran datos provenientes de múltiples fuentes, posiblemente distintas.
}
\end{frame}

\fbckg{negro}
\begin{frame}
\itemized{
\item \textbf{Integrado}: Se integran datos provenientes de múltiples fuentes, posiblemente distintas.
\item \textbf{No volátil}: Una vez almacenados los datos en el DW, la información que éstos representan no debe perderse.
}
\end{frame}

\fbckg{negro}
\begin{frame}
\itemized{
\item \textbf{Integrado}: Se integran datos provenientes de múltiples fuentes, posiblemente distintas.
\item \textbf{No volátil}: Una vez almacenados los datos en el DW, la información que éstos representan no debe perderse.
\item \textbf{Variable en el tiempo}: La información histórica se mantiene en el DW a lo largo del tiempo.
}
\end{frame}

\fbckg{negro}
\begin{frame}
\pointedsl{Arquitectura}
\end{frame}

\fbckg{negro}
\begin{frame}
\begin{center}
\misc{\includegraphics[scale=0.3]{arquitectura}}
\end{center}
\end{frame}

\fbckg{negro}
\begin{frame}
\pointedsl{OLTP vs. DW}
\end{frame}

\fbckg{negro}
\begin{frame}
\misc{\includegraphics[scale=0.27,center]{OltpVSDW3}}
\end{frame}

\fbckg{negro}
\begin{frame}
\pointedsl{Data marts}
\end{frame}

\fbckg{negro}
\begin{frame}
\misc{Los \textbf{Data marts} (DM), son subconjuntos de datos de un data warehouse para áreas específicas.

Entre las características de un data mart destacan:
\begin{itemize}
  \item Usuarios limitados.
  \item Tiene un propósito específico.
  \item Tiene una función de apoyo.
\end{itemize}
}
\end{frame}

\fbckg{negro}
\begin{frame}
\pointedsl{{\Large Metodologías de diseño}}
\end{frame}

\fbckg{negro}
\begin{frame}
\misc{\textbf{Top-Down}: primero se define el DW y luego se desarrollan, construyen y cargan los
DM a partir del mismo.
\newline

\includegraphics[scale=0.16,center]{Top-Down}
}
\end{frame}

\fbckg{negro}
\begin{frame}
\misc{\textbf{Bottom-Up}: se definen previamente los DM y luego se integran
en un DW centralizado.
\newline

\includegraphics[scale=0.16,center]{Bottom-Up}
}
\end{frame}

\fbckg{negro}
\begin{frame}
\pointedsl{{\LARGE Modelo dimensional}}
\end{frame}

\fbckg{negro}
\begin{frame}
\misc{El modelado dimensional es una técnica de diseño lógico de una base de datos, útil para el procesamiento analítico en línea (OLAP), que tiene como ideas centrales el rendimiento y lograr la facilidad de comprensión para el usuario.
\newline

Hay dos conceptos centrales:
\begin{itemize}
  \item Hechos (métricas).
  \item Dimensiones.
\end{itemize}
}

\end{frame}

\fbckg{negro}
\begin{frame}
\pointedsl{{\LARGE Esquema dimensional}}
\end{frame}

\fbckg{negro}
\begin{frame}
\misc{Sabemos que la relación entre todas las tablas de una base de datos se denomina esquema de base de datos. Para un cierto grupo de bases de datos, en las cuales se realizan consultas sobre datos históricos, generalmente se utilizan diseños llamados \textbf{esquemas dimensionales}.
\newline

Un \textbf{esquema dimensional} separa físicamente las medidas que cuantifican el negocio (hechos) de los elementos que los describen (dimensiones).
}
\end{frame}

\fbckg{negro}
\begin{frame}
\misc{\textbf{Esquema estrella}
\newline


En este tipo de esquemas la idea central es tener,


\textbf{Tabla de hechos}

rodeada de

\textbf{Tablas de dimensiones}
}
\end{frame}


\fbckg{negro}
\begin{frame}
\misc{\textbf{Esquema estrella}
%Supongamos que queremos realizar un análisis sobre el importe ganado por cliente, según el producto y para una fecha específica.

\begin{center}
\includegraphics[scale=0.7]{esquema_estrella2}
\end{center}
}
\end{frame}

\fbckg{negro}
\begin{frame}
\misc{\textbf{Esquema estrella}

Este modelo debe estar desnormalizado, es decir que no puede presentarse en tercera forma normal (3ra FN).
Si se normaliza la tabla "PRODUCTOS", obtendremos lo siguiente:

\begin{center}
\includegraphics[scale=0.12]{normalizacion}
\end{center}
}

\end{frame}



\fbckg{negro}
\begin{frame}
\misc{\textbf{Esquema de copo de nieve}
\newline

Este otro tipo de esquemas, es similar al esquema de estrella, salvo que las dimensiones pueden estar conectadas con otras tablas de dimensiones.

Tendremos,


\textbf{Tabla de hechos}

rodeada de 

\textbf{Tablas de dimensiones}

conectadas con

\textbf{Nuevas tablas de dimensiones}
}
\end{frame}

\fbckg{negro}
\begin{frame}
\misc{\textbf{Esquema de copo de nieve}

\begin{center}
\includegraphics[scale=0.5]{copoNieve}
\end{center}
}
\end{frame}

\fbckg{negro}
\begin{frame}
\misc{\textbf{Esquema de constelación}
\newline

Este esquema es una combinación de los dos anteriores, utiliza lo mejor de cada uno, la simplicidad del esquema estrella junto con el cierto nivel de normalización del esquema copo de nieve. Está compuesto de la siguiente forma:

\textbf{Una o más tablas de hechos}

rodeada de

\textbf{Tablas de dimensiones}

conectadas (posiblemente) con

\textbf{Nuevas tablas de dimensiones}
}
\end{frame}

\fbckg{negro}
\begin{frame}
\misc{\textbf{Esquema de constelación}

\begin{center}
\includegraphics[width=8cm,height=8cm,keepaspectratio]{esquema_constelacion}
\end{center}

}
\end{frame}

\fbckg{negro}
\begin{frame}
\pointedsl{{\LARGE Dimensión tiempo}}
\end{frame}

\fbckg{negro}
\begin{frame}
\misc{
Un caso particular entre las dimensiones que se definirán en el data warehouse es el de la \textbf{dimensión de tiempo}.

En un DW, la creación y el mantenimiento de una tabla de dimensión tiempo es obligatoria. Por ello, decimos que un DW es una \textbf{base de datos temporal}.

Tenemos diferentes elecciones para el tipo del campo clave:
\begin{itemize}
  \item Tipo Date.
  \item Tipo Entero: \\
  $\textendash$ Autoincremental (1,2,3,...,35200,...). \\
  $\textendash$ yyyyMMdd (por ejemplo 20150310).
\end{itemize}
}
\end{frame}

\fbckg{negro}
\begin{frame}
\pointedsl{{\LARGE Ventajas/Desventajas}}
\end{frame}

\fbckg{negro}
\begin{frame}
\misc{\textbf{Ventajas:}
\begin{itemize}
  \item Proporciona información clave para toma de decisiones.
  \item Muy útil para el almacenamiento de datos orientados para el análisis, y consultas de datos históricos.
  \item Permite mayor flexibilidad y rapidez en el acceso a la información.
  \item Los DW pueden trabajar en conjunto y, por lo tanto, aumentar el valor operacional de las aplicaciones empresariales.
%   \item Transforma datos en información y la información en conocimientos.
\end{itemize}
}
\end{frame}

\fbckg{negro}
\begin{frame}
\misc{\textbf{Desventajas:}
\begin{itemize}
  \item Requiere continua limpieza, transformación e integración de datos.
  \item A lo largo de su vida los almacenes de datos pueden suponer altos costos. El almacén de datos no suele ser estático. Los costos de mantenimiento son elevados.
  \item Puede presentar ciertas dificultades a la hora de la implementación, debido a los objetivos que pretende la organización.
  \item A veces, ante una petición de información estos devuelven una información subóptima, lo que supone una pérdida para la organización.
\end{itemize}
}
\end{frame}


% \begin{frame}
% \pgfsetfillopacity{1}
% 
% 
% \def\FrameCommand{\fboxsep=0cm \colorbox{\strcolor}} \MakeFramed {\FrameRestore}
% some test text
% some test text
% \endMakeFramed
% 
% \end{frame}


% \fbckg{blank}
% \begin{frame}
% \framedsl{\pitem{pointed slogan} \pitem{framed slogan} \fitem{beamer features}}
% \end{frame}
% 

% \fbckg{blank}
% \begin{frame}
%   \thankyou   %%%% ending slide with thank you notice
% \end{frame}


% \fbckg{blank}
% \begin{frame}
% \sources{
% \includegraphics[scale=0.048]{1} \ flickr/lovelornpoets\\
% \includegraphics[scale=0.2]{2} \ flickr/apsmuseum
% }
% \end{frame}

\end{document}
